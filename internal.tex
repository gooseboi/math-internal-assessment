\documentclass{article}

\usepackage{titling}
\usepackage{amsfonts}

\author{Guzmán Zugnoni}
\title{Non-integer order derivatives}

\renewcommand{\maketitle}{
\begin{center}
	\Huge\textbf \thetitle
\end{center}
}

\setlength{\parindent}{0pt}





\begin{document}
\maketitle

\newpage

\tableofcontents

\newpage

\section{Introduction}

My interest in this topic was born when we learned about the notion of second
and higher order derivatives. We learnt that there was a meaning and an
application of taking a derivative multiple times. The notation used to
represent this operation, however, made me relate it to something else.

To represent the second derivative for instance, one writes $\frac{d^2}{dx^2}$,
which encodes the fact that $\frac{d}{dx}$ was written twice by ``multiplying''
them and writing a $2$ in the exponent. This made me think about
exponentiation, which is the process of repeated multiplication, where
similarly $x \cdot x = x^2$. However, for exponentiation, we learn that it is
possible to exponentiate to non-integral, and non-positive powers, such as
$x^{-1}=\frac{1}{x}$ or $x^\frac{1}{2} = \sqrt{x}$, which have meaning and are
useful in solving mathematic problems, and this led me to think about if there
existed a similar notion regarding derivatives.

My focus in this exploration is going to be to find a way to generalise
higher-order derivatives to all $\mathbb{R}$, and not only $\mathbb{Z}^{+}$.

\section{Negative order derivative}

\section{Cauchy's formula for repeated integration}
\subsection{Proof for n $\in \mathbb{Z}$}
\end{document}
