\documentclass{article}

\usepackage{titling}
\usepackage{amsfonts}

\author{Guzmán Zugnoni}
\title{Non-integer order derivatives}

\renewcommand{\maketitle}{
\begin{center}
	\Huge\textbf \thetitle
\end{center}
}

\setlength{\parindent}{0pt}
\setlength{\parskip}{20pt}

\def\squad{\hskip2\fontdimen3\font}





\begin{document}
\maketitle

\newpage

\tableofcontents

\newpage

\section{Introduction}

My interest in this topic was born when we learned about the notion of second
and higher order derivatives. We learnt that there was a meaning and an
application of taking a derivative multiple times. The notation used to
represent this operation, however, made me relate it to something else.

To represent the second derivative for instance, one writes $\frac{d^2}{dx^2}$,
which encodes the fact that $\frac{d}{dx}$ was written twice by ``multiplying''
them and writing a $2$ in the exponent. This made me think about
exponentiation, which is the process of repeated multiplication, where
similarly $x \cdot x = x^2$. However, for exponentiation, we learn that it is
possible to exponentiate to non-integral, and non-positive powers, such as
$x^{-1}=\frac{1}{x}$ or $x^\frac{1}{2} = \sqrt{x}$, which have meaning and are
useful in solving mathematic problems, and this led me to think about if there
existed a similar notion regarding derivatives.

My focus in this exploration is going to be to find a way to generalise
higher-order derivatives to all $\mathbb{R}$, and not only $\mathbb{Z}^{+}$.

\section{Negative order derivative}

I will follow the same order in which non-integer exponents are first learned,
firstly, exponentiating to negative integers. The expression
$\frac{d^{-1}}{dx^{-1}}$ has no obvious meaning, but it can be interpreted
alike exponents.

Firstly, I will use the notation
$$\frac{d^\alpha}{dx^\alpha}=D^\alpha \quad \forall \squad \alpha \in
\mathbb{R}$$
And for an arbitrary function $f$:
$$\frac{d^\alpha}{dx^\alpha}\left(f)\right. = D^\alpha f$$

By the definition of the second derivative, given an arbitrary differentiable
function $f$: $$D^1D^1 f = D^2 f$$ This can also be extended for the notion of
a third derivative: $$D^1D^2 f = D^3 f$$ It can be extrapolated that, more
generally,
\begin{equation}
	\label{der_sum}
	D^nD^m = D^{n+m} \quad \forall \squad n,m \in \mathbb{Z}^{+}
\end{equation}

This leads to defining negative order derivatives like negative order
exponents, where $x^1 \cdot x^{-1}= x^0$. Similarly, one might say that, by (\ref{der_sum})
\begin{equation}
	\label{inv_eq}
	D^1D^{-1} f = D^0 f
\end{equation}
and this leads to having to define what $D^0$ means.

It makes sense intuitively that taking the derivative of a function 0 times
would leave it without change, but I will check this verifies
algebraically. This is easy to do, as $$D^1 D^0 f = D^{1+0} f = D^1 f$$
Removing the $D^1$ from both sides gives the equation $$D^0 f = f$$ which
proves the intuition that this operation does nothing. This means that the 0th
power of the derivative operator is the identity operator.

Going back to (\ref{inv_eq}), we now see that
\begin{equation}
	\label{inv_eq_simplified}
	D^1D^{-1} f = f
\end{equation}
Since this is true for any arbitrary function $f$, and the order of these two
doesn't matter, i.e. $D^1D^{-1}=D^{-1}D^1$, we see these two operators are
inverses of each other, as composing them results in the identity. Luckily, we
know the inverse to the derivative operator, and that is the integral. To find
the value of $D^{-1}$, we can take the integral of both sides of
(\ref{inv_eq_simplified}), leading to the result
$$
D^{-1} f = \int f dx
$$

So far, negative order derivatives seem simple, as they match our intuition very cleanly:
0th derivatives do nothing, and negative derivatives do the opposite to a derivative.

This matches
an initial guess one may have, which is that a negative
derivative would do the opposite of a positive derivative, but it is important
to verify and formalise this intuition to achieve more accurate results,

\section{Cauchy's formula for repeated integration}
\subsection{Proof for n $\in \mathbb{Z}$}
\end{document}
